\documentclass[a4paper, 12pt]{article}
\usepackage[T2A]{fontenc}
\usepackage[utf8]{inputenc}
\usepackage[english,russian]{babel}
\usepackage{amsmath, amsfonts, amssymb, amsthm, mathtools, misccorr, indentfirst, multirow}
\usepackage{wrapfig}
\usepackage{graphicx}
\usepackage{subfig}
\usepackage{adjustbox}
\usepackage{pgfplots}

\usepackage{geometry}
\geometry{top=20mm}
\geometry{bottom=20mm}
\geometry{left=20mm}
\geometry{right=20mm}
\newcommand{\angstrom}{\textup{\AA}}

\begin{document}
	\begin{enumerate}
		\item Какие особенности в построении оптической схемы для волоконного лазера по сравнению с твердотельным лазером?\par
		\item Каковы методы создания и особенности работы брегговских отражающих элементов в волоконном лазере?\par
			Для практического осуществления режима генерации необходимо ввести положительную обратную связь. В волоконном лазере обратную связь получают отражением излучения на созданных в волокне брегговских решетках на концах волокна. В этом случае электромагнитная волна, распространяющаяся в волоконном волноводе, будет поочередно отражаться от них, усиливаясь при каждом прохождении через активное волокно. Если одну из брегговских решеток, сделать пропускающей часть излучения, то на выходе системы можно выделить пучок полезного излучения.
		\item Как осуществляется накачка в волоконном лазере?\par
			В волоконных лазерах активное волокно имеет сердцевину легированную ионами редкоземельных металлов, внутреннюю оболочку, образующую с сердцевиной волновод, и внешнюю оболочку, образующую волновод с внутренней оболочкой по которому распространяется излучение накачки, введенное в эту область от полупроводникового лазера. Для излучения накачки волновод является многомодовым, в то же время сердцевина активной области образует одномодовый волновод для генерируемого излучения. Для ввода излучения накачки используется несколько методов:
			\begin{enumerate}
				\item Торцевой
				\item Набор V-образных канавок распределенных по боковой поверхности световода
				\item Два светодиода, размещаемых в общей оболочке, один из которых – активный, а в другой вводится излучение накачки, которое в месте их контакта проходит в активную область и осуществляет накачку.
			\end{enumerate}
			Таким образом, осуществляется распределенная накачка активной области.
			\item С чем связана нерегулярность пичков выходного излучения.\par
				Природа пичков до сих пор остается предметом исследований. На основе одномодовой модели лазера можно показать, что регулярные затухающие пульсации связаны с переходными процессами, сопровождающими начало генерации при появлении очередного импульса накачки; иначе говоря, эти пульсации связаны с инерционностью процессов заселения и релаксации уровней. Существенное влияние на характер пичкового режима оказывает многомодовость генерации; в частности, наличие многих мод может вносить в картину пульсаций неупорядоченность.
			\item Рассчитать пороговую и стационарную инверсию в лазере.
			\item Какова природа релаксационных колебаний в лазерах. Чем определяется их характерная частота колебаний.
			\item Чем определяется длительность импульсов при релаксационных колебаниях.
			\item Рассчитать частоту релаксационных колебаний для волоконного лазера, используемого в работе.
			\begin{equation*}
				\omega=\sqrt{\frac{x-1}{\tau_0\tau}};\space x=\frac{W_p}{W_{cp}};\space \frac{1}{\tau_0}=-\frac{c}{2L}\log\left(1-T\right).
			\end{equation*}
			\begin{equation*}
				T=0.8;\space \tau=1200 \text{ мкс}; L=10\text{ см}.
			\end{equation*}
			\begin{equation*}
				\omega=\sqrt{\frac{0.5}{1200\times10^{-6}\times\frac{-20}{3\times10^8}\left(\log0.2\right)^{-1}}}=100294.5\text{ Гц}
			\end{equation*}
			\item Определить время затухания фотонов в резонаторе волоконного лазера.
			\begin{equation*}
				\frac{1}{\tau_0}=-\frac{c}{2L}\log\left(1-T\right)\Rightarrow \tau_0=40\text{ нс}
			\end{equation*}
			\item Вычислить частоту межмодового интервала для продольных мод лазера, используемого в работе.
			\item Определить температуру, при которой работа иттербиевого лазера будет происходить по трехуровневой схеме, если штарковское расщепление уровней рабочего перехода $\sim 500\text{ см}^{-1}$
				\begin{equation*}
					E=\hbar\omega=\hbar c \frac{2\pi}{\lambda}=kT\Rightarrow T=\frac{2\pi\hbar c}{k}=\frac{hc}{\lambda k}
				\end{equation*}
				\begin{equation*}
					W=\frac{1}{\lambda}=\frac{1}{500\text{ см}}\Rightarrow T=\frac{hcW}{k}=720\text{К}
				\end{equation*}
			\item Рассчитать частоту и время затухания релаксационных колебаний для типичного He-Ne, полупроводникового и YAG:Nd$^{3+}$ лазеров.
			\begin{equation*}
				\tau_0=-\frac{2L}{c\cdot\log\left(1-T\right)}
			\end{equation*}
			\begin{equation*}
				\omega=\sqrt{\frac{x-1}{\tau_0\tau}}, t_0=\frac{2\tau}{x}
			\end{equation*}
			\begin{table}[h]
				\centering
				\begin{tabular}{|c|c|c|c|c|c|c|c|}
					\hline
					Лазер & $L$ & $T,\%$ & $\tau$ & $\tau_0$ & $\omega$, рад/с & $t_0$ & $1/\omega$\\
					\hline
					Nd & 1 м & 10 & 230 мкс & 63 нс & $1.9\times 10^5$ & 307 мкс & 33 мкс\\
					He-Ne & 1 м & 0.5 & 100 нс & 1.3 мкс & $1.9\times 10^{6}$ & 0.133 мкс & 3.2 мкс\\
					Полупроводниковый & 0.5 мм & 35 & 1 нс & 7.7 пс & $8\times 10^9$ & 1.3 нс & 781 пкс\\
					\hline
				\end{tabular}
			\end{table}
	\end{enumerate}
\end{document}