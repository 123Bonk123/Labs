\documentclass[a4paper, 12pt]{article}
\usepackage[T2A]{fontenc}
\usepackage[utf8]{inputenc}
\usepackage[english,russian]{babel}
\usepackage{amsmath, amsfonts, amssymb, amsthm, mathtools, misccorr, indentfirst, multirow}
\usepackage{wrapfig}
\usepackage{graphicx}
\usepackage{subfig}
\usepackage{adjustbox}
\usepackage{pgfplots}

\usepackage{geometry}
\geometry{top=20mm}
\geometry{bottom=20mm}
\geometry{left=20mm}
\geometry{right=20mm}
\newcommand{\angstrom}{\textup{\AA}}

\title{Лабораторная работа 8.1\\Определение постоянных Стефана-Больцмана и Планка из анализа теплового излучения накаленного тела}
\author{Нехаев Александр, гр. 654}
\date{\today}
\begin{document}
    \maketitle
    \pagenumbering{gobble}
    \newpage
    \pagenumbering{arabic}
    \tableofcontents
    \newpage
    \section{Введение}
    \paragraph{Цель работы:}
    При помощи модели абсолютно черного тела (АЧТ) проводятся измерения температуры оптическим пирометром с исчезающей нитью и термопарой, исследуется излучение накаленных тел с различной испускательной способностью, определяются постоянные Планка и Стефана-Больцмана.
    \paragraph{В работе используются:}
    оптический пирометр, модель абсолютно черного тела (АЧТ), три исследуемых образца, блок питания, цифровые мультиметры.
    \paragraph{Теоретические основы:}
    Закон Стефана-Больцмана для АЧТ
    \begin{equation}
    	W=\sigma S \left(T^4-T_0^4\right)
    \end{equation}
    Для серого тела, с учетом большой разницы между температурой самого тела и комнатной температуры
    \begin{equation}
    	W=\varepsilon_T\sigma S T^4
    \end{equation}
    где постоянная Стефана-Больцмана определяется из соотношения
    \begin{equation}
    	\sigma=\frac{2\pi^5 k_\text{Б}^4}{15c^2h^3}=5.67\cdot 10^{-12}\frac{\text{Вт}}{\text{см}^2\cdot \text{К}^4}
    \end{equation}
    Выражение для постоянной Планка
    \begin{equation}
    	h=\sqrt[3]{\frac{2\pi^5 k_{\text{Б}}^4}{15c^2\sigma}}
    \end{equation}
    \section{Ход работы}
    \subsection{Изучение работы оптического пирометра}
    \begin{enumerate}
    	\item Включим установку и проверим её функционирование.
    	\item Определим температуру АЧТ при помощи пирометра с учетом постоянной термопары $41\text{мкв}/^\circ \text{C}$ и комнатной температуры $t_k=26^\circ\text{C}$.
    	\begin{table}[h]
    		\label{table1}
    		\centering
    		\begin{tabular}{|c|c|c|c|c|c|}
    			\hline
    			№ & $T_{\text{АЧТ}}$, mV & $T_{\text{АЧТ}}, ^\circ \text{C}$ & $T_{\text{АЧТ}},^\circ C$\\
    			\hline
    			1 & 44.27 & 1079.76 & 1105.76\\
    			2 & 44.49 & 1085.12 & 1111.12\\
    			3 & 44.66 & 1089.27 & 1115.27\\
    			4 & 44.45 & 1084.15 & 1110.15\\
    			5 & 44.31 & 1080.73 & 1106.73\\
    			\hline
    		\end{tabular}
    		\caption{Показания пирометра при измерении температуры модели АЧТ}
    	\end{table}
    	\item С помощью пирометра определим температуру неоновой лампы. Нижний предел измерений пирометра составляет $700^\circ\text{C}$, однако лампа горит более тусклым светом. Отсюда следует, что температура лампы немного ниже $700^\circ\text{C}$. Кроме того, температура лампы достаточно мала, чтобы можно было до неё дотронуться.
    	\item При попытке измерить температуру колец столкнулись со схожей проблемой: их яркость заметно ниже минимальной яркости нити. В связи с этим удалось измерить только температуру керамической трубки: $T_{\text{пир}}=726^\circ\text{C}$.
    \end{enumerate}
    \subsection{Проверка закона Стефано-Больцмана}
    \begin{enumerate}
    	\item Направим пирометр на нить лампы накаливания.
    	\item Снимем зависимость яркостной температуры нити от мощности выделяющейся на ней:
    	\begin{table}[h]
    		\centering
    		\begin{tabular}{|c|c|c|c|c|c|c|c|c|}
    			\hline
    			№ & $T_\text{ярк}$, К & $T_\text{терм}$, K & $I$, А & $U$, В & $W$, Вт & $\sigma W$, Вт & $\log T$ & $\log W$\\
    			\hline
    			1 & 1230. & 1265.08 & 0.8 & 26.65 & 21.32 & 0.05065 & 7.14289 & 3.05965 \\
 				2 & 1363. & 1405.66 & 0.894 & 34.48 & 30.8251 & 0.0613 & 7.24826 & 3.42833 \\
 				3 & 1455. & 1502.91 & 0.962 & 40.44 & 38.9033 & 0.0693 & 7.31516 & 3.66108 \\
 				4 & 1569. & 1623.4 & 1.157 & 58.65 & 67.8581 & 0.09336 & 7.39228 & 4.21742 \\
 				5 & 1718. & 1780.9 & 1.263 & 69.41 & 87.6648 & 0.1073 & 7.48487 & 4.47352 \\
 				6 & 1851. & 1921.48 & 1.368 & 80.4 & 109.987 & 0.12144 & 7.56085 & 4.70036 \\
 				7 & 1952. & 2028.23 & 1.546 & 101.34 & 156.672 & 0.14772 & 7.61492 & 5.05415 \\
 				8 & 2056. & 2138.16 & 1.615 & 109.74 & 177.23 & 0.15819 & 7.6677 & 5.17745 \\
 				9 & 2159. & 2247.03 & 1.735 & 125.32 & 217.43 & 0.17737 & 7.71737 & 5.38188 \\
 				10 & 2195. & 2285.09 & 1.753 & 127.71 & 223.876 & 0.1803 & 7.73416 & 5.41109 \\
 				\hline
    		\end{tabular}
    		\caption{Зависимость яркостной температуры от мощности}
    	\end{table}
    	\item Построим график зависимости $\log W=f\left(\log T\right)$.
    	\begin{figure}
    		\begin{tikzpicture}
    			\begin{axis}[
    				title={$\log W=f(\log T)$},
    				xlabel={$\log T$},
    				ylabel={$\log W$},
    				xmin=7.1,
    				xmax=7.8,
    				ymin=2.9,
    				ymax=5.6,
    				ymajorgrids=true,
    				xmajorgrids=true,
    				grid style=dashed,
    				width=\textwidth,
    			]
    			\addplot+[
    				color=black,
    				mark=square,
    				only marks,
					error bars/.cd,
					y dir=both, y explicit,
					x dir=both, x explicit
    			]
    			coordinates{
    				(7.14289, 3.05965)+-(0.01, 0)
    				(7.24826, 3.42833)+-(0.01, 0)
    				(7.31516, 3.66108)+-(0.01, 0)
    				(7.39228, 4.21742)+-(0.01, 0)
    				(7.48487, 4.47352)+-(0.01, 0)
    				(7.56085, 4.70036)+-(0.01, 0)
    				(7.61492, 5.05415)+-(0.01, 0)
    				(7.6677, 5.17745)+-(0.01, 0)
    				(7.71737, 5.38188)+-(0.01, 0)
    				(7.73416, 5.41109)+-(0.01, 0)
    			};
    			\addplot[
    				domain=7.1:7.8,
    				samples=100,
    				color=black
    			]
    			{-26.0655 + 4.0762*x};
    		\end{axis}
    		\end{tikzpicture}
    		\caption{График зависимости $\log W=f(\log T)$}
    	\end{figure}
    	Таким образом $n=4.08\pm 0.12$.\par
    	\item Определим величину постоянной Стефана-Больцмана и Планка для всех температур больше 1700 K.
    	\begin{table}[h]
    		\centering
    		\begin{tabular}{|c|c|c|c|c|}
    			\hline
    			$T_{\text{терм}}$, К & $\sigma$, $\frac{10^{-7}\text{Вт}}{\text{м}^2\text{K}^4}$ & $\sigma\sigma$, $\frac{10^{-7}\text{Вт}}{\text{м}^2\text{K}^4}$ & $h$, $10^{-34}$ Дж$\cdot$с & $\sigma h$, $10^{-34}$ Дж$\cdot$с\\
    			\hline
    			1780.9 & \text{11.01} & 0.24 & 2.46 & 0.02 \\
 				1921.48 & \text{9.29} & 0.36 & 2.60 & 0.03 \\
 				2028.23 & \text{9.99} & 0.47 & 2.54 & 0.04 \\
 				2138.16 & \text{8.59} & 0.62 & 2.67 & 0.06 \\
				2247.03 & \text{8.15} & 0.80 & 2.72 & 0.09 \\
				2285.09 & \text{7.69} & 0.87 & 2.77 & 0.10 \\
				\hline
    		\end{tabular}
    	\end{table}
    \end{enumerate}
\end{document}
